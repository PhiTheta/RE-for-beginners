\chapter{Trouver des choses importantes/intéressantes dans le code}

Le minimalisme n'est pas une caractéristique prépondérante des logiciels modernes.

\myindex{\Cpp!STL}

Pas parce que les programmeurs écrivent beaucoup, mais parce que de nombreuses bibliothèques
sont couramment liées statiquement aux fichiers exécutable.
Si toutes les bibliothèques externes étaient déplacées dans des fichiers DLL externes,
le monde serait différent. (Une autre raison pour C++ sont la \ac{STL} et autres
bibliothèques templates.)

\newcommand{\FOOTNOTEBOOST}{\footnote{\url{http://go.yurichev.com/17036}}}
\newcommand{\FOOTNOTELIBPNG}{\footnote{\url{http://go.yurichev.com/17037}}}

Ainsi, il est très important de déterminer l'origine de la fonction, si elle provient
d'une bibliothèque standard ou d'une bibliothèque bien connue (comme Boost\FOOTNOTEBOOST,
libpng\FOOTNOTELIBPNG), ou si elle est liée à ce que l'on essaye de trouver dans
le code.

Il est simplement absurde de tout récrire le code en \CCpp pour trouver ce que l'on
cherche.

Une des premières tâches d'un rétro-ingénieur est de trouver rapidement le code dont
il a besoin.

\myindex{\GrepUsage}

Le dés-assembleur \IDA nous permet de chercher parmi les chaînes de texte, les séquences
d'octets et les constantes.
Il est même possible d'exporter le code dans un fichier texte .lst ou .asm et d'utiliser
\TT{grep}, \TT{awk}, etc.

Lorsque vous essayez de comprendre ce que fait un certain code, ceci peut être facile
avec une bibliothèque open-source comme libpng.
Donc, lorsque vous voyez certaines constantes ou chaînes de texte qui vous semblent
familières, il vaut toujours la peine de les \emph{googler}.
Et si vous trouvez le projet open-source où elles sont utilisées, alors il suffit
de comparer les fonctions.
Ceci peut permettre de résoudre certaines parties du problème.

Par exemple, si un programme utilise des fichiers XML, la premières étape peut-être
de déterminer quelle bibliothèque XML est utilisée pour le traitement, puisque les
bibliothèques standards (ou bien connues) sont en général utilisées au lieu de code
fait maison.

\myindex{SAP}
\myindex{Windows!PDB}

Par exemple, j'ai essayé une fois de comprendre comment la compression/décompression
des paquets réseau fonctionne dans SAP 6.0.
C'est un logiciel gigantesque, mais un .\gls{PDB} détaillé avec des informations
de débogage est présent, et c'est pratique.
J'en suis finalement arrivé à l'idée que l'une des fonctions, qui était appelée par
\emph{CsDecomprLZC}, effectuait la décompression des paquets réseau.
Immédiatement, j'ai essayé de googler le nom et rapidement trouvé que la fonction
était utilisée dans MaxDB (c'est un projet open-source de SAP)
\footnote{Plus sur ce sujet dans la section concernée~(\myref{sec:SAPGUI})}.

\url{http://www.google.com/search?q=CsDecomprLZC}

Étonnement, les logiciels MaxDB et SAP 6.0 partagent du code comme ceci pour la compression/
décompression des paquets réseau.

\input{digging_into_code/identification/exec_FR}
% binary files might be also here

\mysection{Communication avec le monde extérieur (niveau fonction)}
Il est souvent recommandé de suivre les arguments de la fonction et sa valeur de
retour dans un débogueur ou \ac{DBI}.
Par exemple, l'auteur a essayé une fois de comprendre la signification d'une fonction
obscure, qui s'est avérée être un tri à bulles mal implémenté\footnote{\url{https://yurichev.com/blog/weird_sort/}}.
(Il fonctionnait correctement, mais plus lentement.)
En même temps, regarder les entrées et sorties de cette fonction aide instantanément
à comprendre ce quelle fait.

Souvent, lorsque vous voyez une division par la multiplication (\myref{sec:divisionbymult}),
mais avez oublié tous les détails du mécanisme, vous pouvez seulement observer l'entrée
et la sortie, et trouver le diviseur rapidement.

% sections:
\input{digging_into_code/communication_win32_FR}
\input{digging_into_code/strings_FR}
\input{digging_into_code/assert_FR}
\input{digging_into_code/constants_FR}
\input{digging_into_code/instructions_FR}
\input{digging_into_code/suspicious_code_FR}
\mysection{Utilisation de nombres magiques lors du tracing}

Souvent, notre but principal est de comprendre comment le programme utilise une valeur
qui a été soit lue d'un fichier ou reçue par le réseau. Le tracing manuel d'une valeur
est souvent une tâche laborieuse. Une des techniques les plus simple pour ceci (bien
que non sûre à 100\%) est d'utiliser votre propre \emph{nombre magique}.

Ceci ressemble à la tomodensitométrie aux rayons X: un agent de radio-contraste est
injecté dans le sang du patient, qui est utilisé pour augmenter la visibilité de
la structure interne du patient aux rayons X.
C'est bien connu comment le sang circule dans les reins d'humains en bonne santé
et si l'agent est dans le sang, il peut être vu facilement en tomographie comment
le sang circule et si il y a des calculs ou des tumeurs.

Nous pouvons prendre un nombre 32-bit comme \TT{0x0badf00d}, ou la date de naissance
de quelqu'un comme \TT{0x11101979} et écrire ce nombre de 4 octets quelque part dans
un fichier utilisé par le programme que nous investiguons.

\myindex{\GrepUsage}
\myindex{tracer}

Puis, en suivant ce programme avec \tracer en mode \emph{code coverage}, avec l'aide
de \emph{grep} ou simplement en cherchant dans le fichier texte (résultant de l'investigation),
nous pouvons facilement voir où la valeur a été utilisée et comment.

Exemple de résultats de \tracer \emph{grepable} en mode \emph{cc}:

\begin{lstlisting}[style=customasmx86]
0x150bf66 (_kziaia+0x14), e=       1 [MOV EBX, [EBP+8]] [EBP+8]=0xf59c934
0x150bf69 (_kziaia+0x17), e=       1 [MOV EDX, [69AEB08h]] [69AEB08h]=0
0x150bf6f (_kziaia+0x1d), e=       1 [FS: MOV EAX, [2Ch]]
0x150bf75 (_kziaia+0x23), e=       1 [MOV ECX, [EAX+EDX*4]] [EAX+EDX*4]=0xf1ac360
0x150bf78 (_kziaia+0x26), e=       1 [MOV [EBP-4], ECX] ECX=0xf1ac360
\end{lstlisting}
% TODO: good example!

Cela peut aussi être utilisé pour des paquets réseau.
Il est important que le \emph{nombre magique} soit unique et ne soit pas présent
dans le code du programme.

\newcommand{\DOSBOXURL}{\href{http://go.yurichev.com/17222}{blog.yurichev.com}}

\myindex{DosBox}
\myindex{MS-DOS}
À part \tracer, DosBox (émulateur MS-DOS) en mode heavydebug est capable d'écrire
de l'information à propos de l'état de tous les registres pour chaque instruction
du programme exécutée dans un fichier texte\footnote{Voir aussi mon article de blog
sur cette fonctionnalité de DosBox: \DOSBOXURL{}}, donc cette technique peut être
utile également pour des programmes DOS.

\input{digging_into_code/loops_FR}
\subsection{Code mal désassemblé}
\label{sec:incorrectly_disasmed_code}

Un rétro ingénieur pratiquant a souvent à faire avec du code mal désassemblé.

\subsubsection{Désassemblage depuis une adresse de début incorrecte (x86)}

Contrairement à ARM et MIPS (où toute instruction a une longueur de 2 ou 4 octets),
les instructions x86 ont une taille variable, donc tout désassembleur démarrant à
une mauvaise adresse qui se trouve au milieu d'une instruction x86 pourra produire
un résultat incorrect.

À titre d'exemple:

\lstinputlisting[style=customasmx86]{\CURPATH/x86_wrong_start_FR.asm}

Il y a des instructions incorrectement désassemblées au début, mais finalement le
désassembleur revient sur la bonne voie.

\subsubsection{À quoi ressemble du bruit aléatoire désassemblé?}

Des propriétés répandues qui peuvent être repérées facilement sont:

\begin{itemize}
\item Dispersion d'instructions inhabituellement grande.
\myindex{x86!\Instructions!PUSH}
\myindex{x86!\Instructions!MOV}
\myindex{x86!\Instructions!CALL}
\myindex{x86!\Instructions!IN}
\myindex{x86!\Instructions!OUT}
Les instructions x86 les plus fréquentes sont \PUSH{}, \MOV{}, \CALL{}, mais ici nous
voyons des instructions de tous les groupes d'instructions: \ac{FPU}, \INS{IN}/\INS{OUT},
instructions systèmes et rares.

\item Valeurs grandes et aléatoires, d'offsets et immédiates.

\item Sauts ayant des offsets incorrects, sautant au milieu d'autres instructions
\end{itemize}

\lstinputlisting[caption=\randomNoise{} (x86),style=customasmx86]{\CURPATH/x86.asm}

\myindex{x86-64}
\lstinputlisting[caption=\randomNoise{} (x86-64),style=customasmx86]{\CURPATH/x64.asm}

\myindex{ARM}
\lstinputlisting[caption=\randomNoise{} (ARM (\ARMMode)),style=customasmARM]{\CURPATH/ARM.asm}

\lstinputlisting[caption=\randomNoise{} (ARM (\ThumbMode)),style=customasmARM]{\CURPATH/ARM_thumb.asm}

\myindex{MIPS}
\lstinputlisting[caption=\randomNoise{} (MIPS little endian),style=customasmMIPS]{\CURPATH/MIPS.asm}

Il est important de garder à l'esprit que du code de dépaquetage et de déchiffrement
construit intelligemment (y compris auto-modifiant) peut avoir l'air aléatoire, mais
s'exécute toujours correctement.
% TODO таких примеров тоже бы добавить


% FIXME comparison!
\subsection{Comparer des \q{snapshots} mémoire}
\label{snapshots_comparing}

La technique consistant à comparer directement deux états mémoire afin de voir les
changements était souvent utilisée pour tricher avec les jeux sur ordinateurs 8-bit
et pour modifier le fichiers des \q{meilleurs scores}.

Par exemple, si vous avez chargé un jeu sur un ordinateur 8-bit (il n'y a pas beaucoup
de mémoire dedans, mais le jeu utilise en général encore moins de mémoire), et que
vous savez que vous avez maintenant, disons, 100 balles, vous pouvez faire un \q{snapshot}
de toute la mémoire et le sauver quelque part. Puis, vous tirez une fois, le compteur
de balles descend à 99, faites un second \q{snapshot} et puis comparer les deux:
il doit y avoir quelque part un octet qui était à 100 au début, et qui est maintenant
à 99.

En considérant le fait que ces jeux 8-bit étaient souvent écrits en langage d'assemblage
et que de telles variables étaient globales, on peut déterminer avec certitude quelle
adresse en mémoire contenait le compteur de balles. Si vous cherchiez toutes les références
à cette adresse dans le code du jeu désassemblé, il n'était pas très difficile de
trouver un morceau de code \glslink{decrement}{décrémentant} le compteur de balles,
puis d'y écrire une, ou plusieurs, instruction \gls{NOP}, et d'avoir un jeu avec
toujours 100 balles.
\myindex{BASIC!POKE}
Les jeux sur ces ordinateurs 8-bit étaient en général chargés à une adresse constante,
aussi, il n'y avait pas beaucoup de versions ce chaque jeu (souvent, une seule version
était répandue pour un long moment), donc les joueurs enthousiastes savaient à quelles
adresses se trouvaient les octets devaient être modifiés (en utilisant l'instruction
BASIC \gls{POKE}) pour le bidouiller. Ceci à conduit à des listes de \q{cheat} qui
contenaient les instructions \gls{POKE} publiées dans des magazines relatifs aux
jeux 8-bit. Voir aussi: \href{http://go.yurichev.com/17114}{Wikipédia}.

\myindex{MS-DOS}

De même, il est facile de modifier le fichier des \q{meilleurs scores}, ceci ne fonctionne
pas seulement avec des jeux 8-bit. Notez votre score et sauvez le fichier quelque part.
Lorsque le décompte des \q{meilleurs scores} devient différent, comparez juste les
deux fichiers, ça peut même être fait avec l'utilitaire DOS FC\footnote{Utilitaire
MS-DOS pour comparer des fichiers binaires.} (les fichiers des \q{meilleurs scores}
sont souvent au format binaire).

Il y aura un endroit où quelques octets seront différents et il est facile de voir
lesquels contiennent le score.
Toutefois, les développeurs de jeux étaient conscient de ces trucs et pouvaient protéger
le programme contre ça.

Exemple quelque peu similaire dans ce livre: \myref{Millenium_DOS_game}.

% TODO: пример с какой-то простой игрушкой?

\subsubsection{Une histoire vraie de 1999}

\myindex{ICQ}
C'était un temps de l'engouement pour la messagerie ICQ, au moins dans les pays de
l'ex-URSS.
Cette messagerie avait une particularité --- certains utilisateurs ne voulaient pas
partager leur état en ligne avec tout le monde.
Et vous deviez demander une \emph{autorisation} à cet utilisateur.
Il pouvait vous autoriser à voir son état, ou pas.

Voici ce que j'ai fait:

\begin{itemize}
\item Ajouté un utilisateur.
\item Un utiliseur est apparu dans la liste de contact, dans la section ``attente d'autorisation''.
\item Fermé ICQ.
\item Sauvegardé la base de données ICQ.
\item Ouvert à nouveau ICQ.
\item L'utilisateur m'a \emph{autorisé}.
\item Refermé ICQ et comparé les deux base de données.
\end{itemize}

Il s'est avéré que: les deux bases de données ne différaient que d'un octet.
Dans la première version: \verb|RESU\x03|, dans la seconde: \verb|RESU\x02|.
(``RESU'', signifie probablement ``USER'', i.e., un entête d'une structure où toutes
les informations à propos d'un utilisateur étaient stockées.)
Cela signifie que l'information sur l'autorisation n'était pas stockée sur le serveur,
mais sur le client. Vraisemblablement, la valeur 2/3 reflétait l'état de l'\q{autorisation}.

\subsubsection{Registres de Windows}

Il est aussi possible de comparer les registres de Windows avant et après l'installation
d'un programme.

C'est une méthode courante  que de trouver quels sont les éléments des registres
utilisés par le programme. Peut-être que ceci est la raison pour laquelle le shareware
de \q{nettoyage des registres windows} est si apprécié.

À propos, voici comment sauver les registres de Windows dans des fichiers texte:

\begin{lstlisting}
reg export HKLM HKLM.reg
reg export HKCU HKCU.reg
reg export HKCR HKCR.reg
reg export HKU HKU.reg
reg export HKCC HKCC.reg
\end{lstlisting}

\myindex{UNIX!diff}
Ils peuvent être comparés en utilisant diff...

\subsubsection{Comparateur à clignotement}

La comparaison de fichiers ou d'images mémoire nous rappelle le comparateur à clignotement
\footnote{\url{http://go.yurichev.com/17348}}:
Un dispositif utilisé autrefois par les astronomes pour trouver les objets célestes
changeant de position.

Les comparateurs à clignotement permet d'alterner rapidement entre deux photographies
prisent à des moments différents, de façon à faire apparaître les différences visuellement.

À propos, Pluton a été découverte avec un comparateur à clignotement en 1930.

\mysection{Détection de l'\ac{ISA}}
\label{ISA_detect}

Souvent, vous avez à faire à un binaire avec un \ac{ISA} inconnu.
Peut-être que la manière la plus facile de détecter l'\ac{ISA} est d'en essayer plusieurs
dans \IDA, objdump ou un autre désassembleur.

Pour réussir ceci, il faut comprendre la différence entre du code incorrectement
et celui correctement désassemblé.

% subsection:
\renewcommand{\CURPATH}{digging_into_code/incorrect_disassembly}
\subsection{Code mal désassemblé}
\label{sec:incorrectly_disasmed_code}

Un rétro ingénieur pratiquant a souvent à faire avec du code mal désassemblé.

\subsubsection{Désassemblage depuis une adresse de début incorrecte (x86)}

Contrairement à ARM et MIPS (où toute instruction a une longueur de 2 ou 4 octets),
les instructions x86 ont une taille variable, donc tout désassembleur démarrant à
une mauvaise adresse qui se trouve au milieu d'une instruction x86 pourra produire
un résultat incorrect.

À titre d'exemple:

\lstinputlisting[style=customasmx86]{\CURPATH/x86_wrong_start_FR.asm}

Il y a des instructions incorrectement désassemblées au début, mais finalement le
désassembleur revient sur la bonne voie.

\subsubsection{À quoi ressemble du bruit aléatoire désassemblé?}

Des propriétés répandues qui peuvent être repérées facilement sont:

\begin{itemize}
\item Dispersion d'instructions inhabituellement grande.
\myindex{x86!\Instructions!PUSH}
\myindex{x86!\Instructions!MOV}
\myindex{x86!\Instructions!CALL}
\myindex{x86!\Instructions!IN}
\myindex{x86!\Instructions!OUT}
Les instructions x86 les plus fréquentes sont \PUSH{}, \MOV{}, \CALL{}, mais ici nous
voyons des instructions de tous les groupes d'instructions: \ac{FPU}, \INS{IN}/\INS{OUT},
instructions systèmes et rares.

\item Valeurs grandes et aléatoires, d'offsets et immédiates.

\item Sauts ayant des offsets incorrects, sautant au milieu d'autres instructions
\end{itemize}

\lstinputlisting[caption=\randomNoise{} (x86),style=customasmx86]{\CURPATH/x86.asm}

\myindex{x86-64}
\lstinputlisting[caption=\randomNoise{} (x86-64),style=customasmx86]{\CURPATH/x64.asm}

\myindex{ARM}
\lstinputlisting[caption=\randomNoise{} (ARM (\ARMMode)),style=customasmARM]{\CURPATH/ARM.asm}

\lstinputlisting[caption=\randomNoise{} (ARM (\ThumbMode)),style=customasmARM]{\CURPATH/ARM_thumb.asm}

\myindex{MIPS}
\lstinputlisting[caption=\randomNoise{} (MIPS little endian),style=customasmMIPS]{\CURPATH/MIPS.asm}

Il est important de garder à l'esprit que du code de dépaquetage et de déchiffrement
construit intelligemment (y compris auto-modifiant) peut avoir l'air aléatoire, mais
s'exécute toujours correctement.
% TODO таких примеров тоже бы добавить



\subsection{Code désassemblé correctement}
\label{correctly_disasmed_code}

Chaque \ac{ISA} a une douzaines d'instructions les plus utilisées, toutes les autres
le sont beaucoup moins souvent.

Concernant le x86, il est intéressant de savoir le fait que les instructions d'appel
de fonctions (\PUSH/\CALL/\ADD) et \MOV sont les morceaux de code les plus fréquemment
exécutées dans presque tous les programmes que nous utilisons.
Autrement dit, le \ac{CPU} est très occupé à passer de l'information entre les niveaux
d'abstraction, ou, on peut dire qu'il est très occupé à commuter entre ces niveaux.
Indépendamment du type d'\ac{ISA}.
Ceci a un coût de diviser les problèmes entre plusieurs niveaux d'abstraction (ainsi
les êtres humain peuvent travailler plus facilement avec).



\mysection{Autres choses}

\subsection{Idée générale}

Un rétro-ingénieur doit essayer se se mettre dans la peau d'un programmeur aussi
souvent que possible.
Pour adopter son point de vue et se demander comment il aurait résolu des tâches
d'un cas spécifique.

\subsection{Ordre des fonctions dans le code binaire}

Toutes les fonctions situées dans un unique fichier .c ou .cpp sont compilées dans
le fichier objet (.o) correspondant.
Plus tard, l'éditeur de liens mets tous les fichiers dont il a besoin ensemble, sans
changer l'ordre ni les fonctions.
Par conséquent, si vous voyez deux ou plus fonctions consécutives, cela signifie
qu'elles étaient situées dans le même fichier source (à moins que vous ne soyez en
limite de deux fichiers objet, bien sûr).
Ceci signifie que ces fonctions ont quelque chose en commun, qu'elles sont des fonctions
du même niveau d'\ac{API}, de la même bibliothèque, etc.

\myindex{CryptoPP}
Ceci est une histoire vraie de pratique: il était une fois, alors que je cherchais
des fonctions relatives à Twofish dans un programme lié à la bibliothèque CryptoPP,
en particulier des fonctions de chiffrement/déchiffrement.\\
J'ai trouvé la fonction \verb|Twofish::Base::UncheckedSetKey()| mais pas d'autres.
Après avoir cherché dans le code source
\verb|twofish.cpp|\footnote{\url{https://github.com/weidai11/cryptopp/blob/b613522794a7633aa2bd81932a98a0b0a51bc04f/twofish.cpp}},
il devint clair que toutes les fonctions étaient situées dans ce module (\verb|twofish.cpp|).\\
Donc j'ai essayé toutes les fonctions qui suivaient \verb|Twofish::Base::UncheckedSetKey()|---comme elles arrivaient,\\
une a été \verb|Twofish::Enc::ProcessAndXorBlock()|, une autre---\verb|Twofish::Dec::ProcessAndXorBlock()|.

\subsection{Fonctions minuscules}

Les fonctions minuscules comme les fonctions vides (\myref{empty_func})
ou les fonctions qui renvoient juste ``true'' (1) ou ``false'' (0) (\myref{ret_val_func})
sont très communes, et presque tous les compilateurs corrects tendent à ne mettre
qu'une seule fonction de ce genre dans le code de l'exécutable résultant, même si
il y avait plusieurs fonctions similaires dans le code source.
Donc, à chaque fois que vous voyez une fonction minuscule consistant seulement en
\TT{mov eax, 1 / ret} qui est référencée (et peut être appelée) dans plusieurs endroits
qui ne semblent pas reliés les uns au autres, ceci peut résulter d'une telle optimisation.%

\subsection{\Cpp}

Les données \ac{RTTI}~(\myref{RTTI})- peuvent être utiles pour l'identification des
classes \Cpp.

%\subsection{Crash on purpose}
\subsection{Crash délibéré}

Souvent, vous voulez savoir quelle fonction a été exécutée, et laquelle ne l'a pas
été.
Vous pouvez utiliser un débogueur, mais sur des architectures exotiques, il peut
ne pas en avoir, donc la façon la plus simple est d'y mettre un opcode invalide,
ou quelque chose comme \INS{INT3} (0xCC).
Le crash signalera le fait que l'instruction a été exécutée.

Un autre exemple de crash délibéré: \myref{dmalloc_KILL_PROCESS}.

