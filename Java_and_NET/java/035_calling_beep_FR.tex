% TODO proof-reading
\subsection{Appel de beep()}

Ceci est un simple appel de deux fonctions sans paramètre:


\begin{lstlisting}[style=customjava]
	public static void main(String[] args)
	{
		java.awt.Toolkit.getDefaultToolkit().beep();
	};
\end{lstlisting}

\begin{lstlisting}
  public static void main(java.lang.String[]);
    flags: ACC_PUBLIC, ACC_STATIC
    Code:
      stack=1, locals=1, args_size=1
         0: invokestatic  #2      // Method java/awt/Toolkit.getDefaultToolkit:()Ljava/awt/Toolkit;
         3: invokevirtual #3      // Method java/awt/Toolkit.beep:()V
         6: return        
\end{lstlisting}

Le premier \TT{invokestatic} à l'offset 0 appelle\\
\TT{java.awt.Toolkit.getDefaultToolkit()},
qui renvoie une référence sur un objet de la classe \TT{Toolkit}.\\
L'instruction \TT{invokevirtual} à l'offset 3 appelle la méthode \TT{beep()} de cette
classe.

