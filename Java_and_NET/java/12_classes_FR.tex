% TODO proof-reading
\subsection{Classes}

Classe simple:

\begin{lstlisting}[caption=test.java,style=customjava]
public class test
{
        public static int a;
        private static int b;

        public test()
        {
            a=0;
            b=0;
        }
        public static void set_a (int input)
	{
		a=input;
	}
	public static int get_a ()
	{
		return a;
	}
	public static void set_b (int input)
	{
		b=input;
	}
	public static int get_b ()
	{
		return b;
	}
}
\end{lstlisting}

Le constructeur met simplement les deux champs à zéro:


\begin{lstlisting}
  public test();
    flags: ACC_PUBLIC
    Code:
      stack=1, locals=1, args_size=1
         0: aload_0
         1: invokespecial #1         // Method java/lang/Object."<init>":()V
         4: iconst_0
         5: putstatic     #2         // Field a:I
         8: iconst_0
         9: putstatic     #3         // Field b:I
        12: return
\end{lstlisting}

Setter de \TT{a}:

\begin{lstlisting}
  public static void set_a(int);
    flags: ACC_PUBLIC, ACC_STATIC
    Code:
      stack=1, locals=1, args_size=1
         0: iload_0
         1: putstatic     #2         // Field a:I
         4: return
\end{lstlisting}

Getter de \TT{a}:

\begin{lstlisting}
  public static int get_a();
    flags: ACC_PUBLIC, ACC_STATIC
    Code:
      stack=1, locals=0, args_size=0
         0: getstatic     #2         // Field a:I
         3: ireturn
\end{lstlisting}

Setter de \TT{b}:

\begin{lstlisting}
  public static void set_b(int);
    flags: ACC_PUBLIC, ACC_STATIC
    Code:
      stack=1, locals=1, args_size=1
         0: iload_0
         1: putstatic     #3         // Field b:I
         4: return
\end{lstlisting}

Getter de \TT{b}:

\begin{lstlisting}
  public static int get_b();
    flags: ACC_PUBLIC, ACC_STATIC
    Code:
      stack=1, locals=0, args_size=0
         0: getstatic     #3         // Field b:I
         3: ireturn
\end{lstlisting}

Il n'y a aucune différence dans le code qui fonctionne avec des champs publics ou
privés.

Mais ce type d'information est présent dans le fichier \TT{.class} et il n'est pas
possible d'accéder aux champs privés depuis n'importe où.


Créons un objet et appelons sa méthode:


\begin{lstlisting}[caption=ex1.java,style=customjava]
public class ex1
{
	public static void main(String[] args)
	{
		test obj=new test();
		obj.set_a (1234);
		System.out.println(obj.a);
	}
}
\end{lstlisting}

\begin{lstlisting}
  public static void main(java.lang.String[]);
    flags: ACC_PUBLIC, ACC_STATIC
    Code:
      stack=2, locals=2, args_size=1
         0: new           #2        // class test
         3: dup
         4: invokespecial #3        // Method test."<init>":()V
         7: astore_1
         8: aload_1
         9: pop
        10: sipush        1234
        13: invokestatic  #4        // Method test.set_a:(I)V
        16: getstatic     #5        // Field java/lang/System.out:Ljava/io/PrintStream;
        19: aload_1
        20: pop
        21: getstatic     #6        // Field test.a:I
        24: invokevirtual #7        // Method java/io/PrintStream.println:(I)V
        27: return
\end{lstlisting}

L'instruction \TT{new} crée un objet, mais n'appelle pas le constructeur (il est
appelé à l'offset 4).

La méthode \TT{set\_a()} est appelée à l'offset 16.

Le champ \TT{a} est accédé en utilisant l'instruction \TT{getstatic} à l'offset 21.

