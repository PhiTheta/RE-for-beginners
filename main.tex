\documentclass[a4paper,oneside]{book}

% http://www.tex.ac.uk/FAQ-noroom.html
\usepackage{etex}

\usepackage[table,usenames,dvipsnames]{xcolor}

\usepackage{fontspec}
% fonts
%\setmonofont{DroidSansMono}
%\setmainfont[Ligatures=TeX]{PT Sans}
%\setmainfont{DroidSans}
\setmainfont{DejaVu Sans}
\setmonofont{DejaVu Sans Mono}
\usepackage{polyglossia}
\defaultfontfeatures{Scale=MatchLowercase} % ensure all fonts have the same 1ex
\usepackage{ucharclasses}
\usepackage{csquotes}

\ifdefined\ENGLISH
%\wlog{main ENGLISH defined OK}
\setmainlanguage{english}
\setotherlanguage{russian}
\fi

\ifdefined\RUSSIAN
\setmainlanguage{russian}
%\newfontfamily\cyrillicfont{LiberationSans}
%\newfontfamily\cyrillicfonttt{LiberationMono}
%\newfontfamily\cyrillicfontsf{lmsans10-regular.otf}
\setotherlanguage{english}
\fi

\ifdefined\GERMAN
%\wlog{main GERMAN defined OK}
\setmainlanguage{german}
\setotherlanguage{english}
\fi

\ifdefined\SPANISH
\setmainlanguage{spanish}
\setotherlanguage{english}
\fi

\ifdefined\ITALIAN
\setmainlanguage{italian}
\setotherlanguage{english}
\fi

\ifdefined\BRAZILIAN
\setmainlanguage{portuges}
\setotherlanguage{english}
\fi

\ifdefined\POLISH
\setmainlanguage{polish}
\setotherlanguage{english}
\fi

\ifdefined\DUTCH
\setmainlanguage{dutch}
\setotherlanguage{english}
\fi

\ifdefined\TURKISH
\setmainlanguage{turkish}
\setotherlanguage{english}
\fi

\ifdefined\THAI
\setmainlanguage{thai}
%\usepackage[thai]{babel}
%\usepackage{fonts-tlwg}
\setmainfont[Script=Thai]{TH SarabunPSK}
\newfontfamily{\thaifont}[Script=Thai]{TH SarabunPSK}
\let\thaifonttt\ttfamily
\setotherlanguage{english}
\fi

\ifdefined\FRENCH
\setmainlanguage{french}
\setotherlanguage{english}
\fi

\ifdefined\JAPANESE
\usepackage{xeCJK}
\xeCJKallowbreakbetweenpuncts
\defaultfontfeatures{Ligatures=TeX,Scale=MatchLowercase}
\setCJKmainfont{IPAGothic}
\setCJKsansfont{IPAGothic}
\setCJKmonofont{IPAGothic}
\DeclareQuoteStyle{japanese}
  {「}
  {」}
  {『}
  {』}
\setquotestyle{japanese}
\setmainlanguage{japanese}
\setotherlanguage{english}
\fi

\usepackage{microtype}
\usepackage{fancyhdr}
\usepackage{listings}
%\usepackage{ulem} % used for \sout{}...
\usepackage{url}
\usepackage{graphicx}
\usepackage{makeidx}
\usepackage[cm]{fullpage}
%\usepackage{color}
\usepackage{fancyvrb}
\usepackage{xspace}
\usepackage{tabularx}
\usepackage{framed}
\usepackage{parskip}
\usepackage{epigraph}
\usepackage{ccicons}
\usepackage[nottoc]{tocbibind}
\usepackage{longtable}
\usepackage[footnote,printonlyused,withpage]{acronym}
\usepackage[]{bookmark,hyperref} % must be last
\usepackage[official]{eurosym}
\usepackage[usestackEOL]{stackengine}

% ************** myref
% http://tex.stackexchange.com/questions/228286/how-to-mix-ref-and-pageref#228292
\ifdefined\RUSSIAN
\newcommand{\myref}[1]{%
  \ref{#1}
  (стр.~\pageref{#1})%
  }
% FIXME: I wasn't able to force varioref to output russian text...
\else
\usepackage{varioref}
\newcommand{\myref}[1]{\vref{#1}}
\fi
% ************** myref

\usepackage{glossaries}
\usepackage{tikz}
%\usepackage{fixltx2e}
\usepackage{bytefield}

\usepackage{amsmath}
\usepackage{MnSymbol}
\undef\mathdollar

\usepackage{float}

\usepackage{shorttoc}
\usetikzlibrary{calc,positioning,chains,arrows}
\usepackage[margin=0.5in,headheight=15.5pt]{geometry}
% FIXME would be good without "margin="...
%\usepackage[headheight=15.5pt]{geometry}

%--------------------
% to prevent clashing of numbers and titles in TOC:
% https://tex.stackexchange.com/a/64124
\usepackage{tocloft}% http://ctan.org/pkg/tocloft
\makeatletter
\renewcommand{\numberline}[1]{%
  \@cftbsnum #1\@cftasnum~\@cftasnumb%
}
\makeatother
%--------------------

\newcommand{\footnoteref}[1]{\textsuperscript{\ref{#1}}}

%\definecolor{lstbgcolor}{rgb}{0.94,0.94,0.94}

% I don't know why this voodoo works, but without all-caps, it can't find LIGHT-GRAY color. WTF?
% see also: https://tex.stackexchange.com/questions/64298/error-with-xcolor-package
\definecolor{light-gray}{gray}{0.87}
\definecolor{LIGHT-GRAY}{gray}{0.87}
\definecolor{RED}{rgb}{1,0,0}
\makeindex

\input{macros}
\EN{\input{macro_lang_EN}}
\RU{\input{macro_lang_RU}}
\FR{\newcommand{\AcronymsUsed}{Acronymes utilisés}

\newcommand{\TitleRE}{Rétro-ingénierie pour Débutants}

\newcommand{\TitleUAL}{Comprendre le langage d'assemblage}

\newcommand{\AUTHOR}{Dennis Yurichev}

\newcommand{\figname}{fig.\xspace}
\newcommand{\listingname}{listado.\xspace}
% FIXME get rid of:
\newcommand{\bitsENRU}{bits\xspace}
\newcommand{\Sourcecode}{Code source\xspace}
\newcommand{\Seealso}{Voir également\xspace}
\newcommand{\tableheader}{\headercolor{} offset & \headercolor{} description }
% instructions descriptions
\newcommand{\ASRdesc}{décalage arithmétique vers la droite}

% x86 registers tables
\newcommand{\RegHeaderTop}{ \multicolumn{8}{ | c | }{ Octet d'indice } }
% TODO: non-overlapping color!
\newcommand{\RegHeader}{7 & 6 & 5 & 4 & 3 & 2 & 1 & 0}

\newcommand{\ReturnAddress}{Adresse de retour}

\newcommand{\localVariable}{variable locale}

\newcommand{\savedValueOf}{valeur enregistrée de}

% for index
\newcommand{\GrepUsage}{Utilisation de grep}
\newcommand{\SyntacticSugar}{Sucre syntaxique}
\newcommand{\CompilerAnomaly}{Anomalies du compilateur}
\newcommand{\CLanguageElements}{Éléments du langage C}
\newcommand{\CStandardLibrary}{Bibliothèque standard C}
\newcommand{\Instructions}{Instructions}
\newcommand{\Pseudoinstructions}{Pseudo-instructions}
\newcommand{\Prefixes}{Préfixes}

\newcommand{\Flags}{Flags}
\newcommand{\Registers}{Registres}
\newcommand{\registers}{registres}
\newcommand{\Stack}{Pile}
\newcommand{\Recursion}{Récursivité}
\newcommand{\RAM}{RAM}
\newcommand{\ROM}{ROM}
\newcommand{\Pointers}{Pointeurs}
\newcommand{\BufferOverflow}{Débordement de tampon}

\newcommand{\Conclusion}{Conclusion}

\newcommand{\Exercise}{Exercice}
\newcommand{\Exercises}{Exercices\xspace}
\newcommand{\Arrays}{Tableaux}
\newcommand{\Cpp}{C++\xspace}
\newcommand{\CCpp}{C/C++\xspace}
\newcommand{\NonOptimizing}{sans optimisation\xspace}
\newcommand{\Optimizing}{avec optimisation\xspace}
\newcommand{\ARMMode}{Mode ARM\xspace}
\newcommand{\ThumbMode}{Mode Thumb\xspace}
\newcommand{\ThumbTwoMode}{Mode Thumb-2\xspace}
\newcommand{\AndENRU}{et\xspace}
\newcommand{\OrENRU}{ou\xspace}
\newcommand{\InENRU}{dans\xspace}
\newcommand{\ForENRU}{pour\xspace}
\newcommand{\LineENRU}{ligne\xspace}

\newcommand{\DataProcessingInstructionsFootNote}{Ces instructions sont également appelées \q{instructions de traitement de données}}

% for .bib files
\newcommand{\AlsoAvailableAs}{Aussi disponible en\xspace}

% section names
\newcommand{\ShiftsSectionName}{Décalages}
\newcommand{\SignedNumbersSectionName}{Représentations des nombres signés}
\newcommand{\HelloWorldSectionName}{Hello, world!}
\newcommand{\SwitchCaseDefaultSectionName}{switch()/case/default}
\newcommand{\PrintfSeveralArgumentsSectionName}{printf() avec plusieurs arguments}
\newcommand{\BitfieldsChapter}{Manipulation de bits spécifiques}
\newcommand{\ArithOptimizations}{Remplacement de certaines instructions arithmétiques par d'autres}

\newcommand{\FPUChapterName}{Unité à virgule flottante}
\newcommand{\MoreAboutStrings}{Plus d'information sur les chaînes}
\newcommand{\DivisionByMultSectionName}{Division par la multiplication}
\newcommand{\Answer}{Réponse}
\newcommand{\WhatThisCodeDoes}{Que fait ce code ?}
\newcommand{\WorkingWithFloatAsWithStructSubSubSectionName}{Travailler avec le type float comme une structure}

\newcommand{\MinesweeperWinXPExampleChapterName}{Démineur (Windows XP)}
\newcommand{\StructurePackingSectionName}{Organisation des champs dans la structure}
\newcommand{\StructuresChapterName}{Structures}
\newcommand{\PICcode}{code indépendant de la position}
\newcommand{\CapitalPICcode}{Code indépendant de la position}
\newcommand{\Loops}{Boucles}

% C
\newcommand{\PostIncrement}{Post-incrémentation}
\newcommand{\PostDecrement}{Post-décrémentation}
\newcommand{\PreIncrement}{Pré-incrémentation}
\newcommand{\PreDecrement}{Pré-décrémentation}

% MIPS
\newcommand{\GlobalPointer}{Pointeur Global}

\newcommand{\garbage}{déchets}
\newcommand{\IntelSyntax}{Syntaxe Intel}
\newcommand{\ATTSyntax}{Syntaxe AT\&T}
\newcommand{\randomNoise}{bruit aléatoire}
\newcommand{\Example}{Exemple}
\newcommand{\argument}{argument}
\newcommand{\MarkedInIDAAs}{marqué dans \IDA comme}
\newcommand{\stepover}{enjamber}
\newcommand{\ShortHotKeyCheatsheet}{Anti-sèche des touches de raccourci}

\newcommand{\assemblyOutput}{résultat en sortie de l'assembleur}
% was in common_URLS.tex:
\newcommand{\URLWPDA}{\href{https://fr.wikipedia.org/wiki/Alignement_en_m\%C3\%A9moire}{Wikipedia: Alignement en mémoire}}

% ML prefix is for multi-lingual words and sentences:
\newcommand{\MLHeap}{Heap}
\newcommand{\MLStack}{Pile}
\newcommand{\MLStackOverflow}{Débordement de pile}
\newcommand{\MLStartOfHeap}{Début du heap}
\newcommand{\MLStartOfStack}{Début de la pile}
\newcommand{\MLinputA}{entrée A}
\newcommand{\MLinputB}{entrée B}
\newcommand{\MLoutput}{sortie}
\newcommand{\SoftwareCracking}{cracking de logiciel}

}
\DE{\input{macro_lang_DE}}
\JA{\input{macro_lang_JP}}
\IT{\input{macro_lang_IT}}
\PL{\input{macro_lang_PL}}
\ES{\input{macro_lang_ES}}
\NL{\input{macro_lang_NL}}
\TR{\input{macro_lang_TR}}
\PTBR{\input{macro_lang_PTBR}}

\ifdefined\UAL
\newcommand{\TitleMain}{\TitleUAL}
\newcommand{\TitleAux}{\TitleRE}
\else
\newcommand{\TitleMain}{\TitleRE}
\newcommand{\TitleAux}{\TitleUAL}
\fi

\EN{\input{glossary_EN}}
\RU{\input{glossary_RU}}
\FR{\input{glossary_FR}}
\DE{\input{glossary_DE}}
\IT{\input{glossary_IT}}
\JA{\input{glossary_JA}}

\makeglossaries

\hypersetup{
    colorlinks=true,
    allcolors=blue,
    pdfauthor={\AUTHOR},
    pdftitle={\TitleMain}
    }

%\ifdefined\RUSSIAN
\newcommand{\LstStyle}{\ttfamily\small}
%\else
%\newcommand{\LstStyle}{\ttfamily}
%\fi

% inspired by http://prismjs.com/
\definecolor{digits}{RGB}{0,0,0}
\definecolor{bg}{RGB}{255,255,255}
%\definecolor{bg}{RGB}{255,252,250}
\definecolor{col1}{RGB}{154,20,150}
\definecolor{col2}{RGB}{112,128,144}
\definecolor{col3}{RGB}{10,120,180}
\definecolor{col4}{RGB}{106,164,108}

\lstset{
    %backgroundcolor=\color{lstbgcolor},
    %backgroundcolor=\color{light-gray},
    backgroundcolor=\color{bg},
    basicstyle=\LstStyle,
    breaklines=true,
    %prebreak=\raisebox{0ex}[0ex][0ex]{->},
    %postbreak=\raisebox{0ex}[0ex][0ex]{->},
    prebreak=\raisebox{0ex}[0ex][0ex]{\ensuremath{\rhookswarrow}},
    postbreak=\raisebox{0ex}[0ex][0ex]{\ensuremath{\rcurvearrowse\space}},
    frame=single,
    columns=fullflexible,keepspaces,
    escapeinside=§§,
    inputencoding=utf8
}

\input{syntax_color}

\ifdefined\RUSSIAN
\renewcommand\lstlistingname{Листинг}
\renewcommand\lstlistlistingname{Листинг}
\fi

\DeclareMathSizes{12}{30}{16}{12}%

% see also:
% http://tex.stackexchange.com/questions/129225/how-can-i-get-get-makeindex-to-ignore-capital-letters
% http://tex.stackexchange.com/questions/18336/correct-sorting-of-index-entries-containing-macros
\def\myindex#1{\expandafter\index\expandafter{#1}}

\begin{document}

\iffalse
% fancyhdr =============================================================================================================================
\pagestyle{fancy}
\setlength{\headheight}{13pt}
% https://tex.stackexchange.com/questions/10043/page-number-position
\fancyhf{}
\fancyhead[R]{\thepage} % suppress chapter name, add page number (upper right corner)

\ifdefined\ENGLISH
\cfoot{\small (For statistics collecting purposes) if you've read this far, please click \href{https://beginners.re/stat/EN \today p.\thepage}{here}. Thanks! \normalsize}
% https://tex.stackexchange.com/questions/13406/how-to-add-a-horizontal-line-above-the-footer-with-fancyhdr
\renewcommand{\footrulewidth}{0.4pt}
\fi

\ifdefined\RUSSIAN
\cfoot{\small (В целях сбора статистики) если вы дочитали до этого места, пожалуйста, нажмите \href{https://beginners.re/stat/RU \today p.\thepage}{здесь}. Спасибо! \normalsize}
% https://tex.stackexchange.com/questions/13406/how-to-add-a-horizontal-line-above-the-footer-with-fancyhdr
\renewcommand{\footrulewidth}{0.4pt}
\fi
% ======================================================================================================================================
\fi

\VerbatimFootnotes

\frontmatter

\RU{% To translators: don't bother to translate this... english-only version.

\begin{center}
\LARGE{} Это моя собственная доска объявления \normalsize{}
\end{center}

\textbf{Эта книга наверняка уже устарела}.
(Если только не была скачана прямо сейчас с \url{https://beginners.re/}.)

Книга \href{https://github.com/DennisYurichev/RE-for-beginners/commits/master}{меняется очень часто},
контент добавляется, ошибки (будем надеяться) исправляются.
Также, в первую очередь книга пишется на английском, а перевод на русский немного запаздывает.
Последняя версия всегда на \url{https://beginners.re/}.

А PDF-файл, который вы сейчас читаете, был скомпилирован \today{}.

\myhrule{}

Если вы распечатали эту книгу на бумаге, не могли бы вы прислать мне её фотографию, для коллекции?\\
\EMAIL{}, Telegram: @yurichev.

\myhrule{}

Мои дорогие читатели! Время от времени, у меня появляются вопросы, и я не знаю, кого (или где) спросить.
Или я просто ленив...
Поможете мне?

\myhrule{}

Есть комп с дешевыми кулерами, в старом, дешевом и раздолбанном кузове.
Linux раз в 1-2 секунды меняет скорость вращения кулера процессора, но немного.
Например, между 833 RPM и 834 RPM.
И кулера и кузов дешевые, они друг с другом, видимо, немного резонируют,
поэтому от таких изменений меняется тон гула/шума, раз в 1-2 секунды, что дико раздражает.
Как заставить Linux менять скорость реже, например один раз в 10-20 секунд?
\verb|INTERVAL=10| в \verb|/etc/fancontrol| не помогает.

Тут конфиг и вывод sensors: \url{https://pastebin.com/yk7EuiBL}.

Мать: Intel Desktop Board DZ77SL-50K.

\myhrule{}

Кто-нибудь может мне помочь с CBMC? Есть вопросы...

\myhrule{}

В самом начале 1990-х вышла книга на русском, где были законы Мерфи, Паркинсона, итд...
Как называлась?
Забыл.

\myhrule{}

Какой меломанский HiFi mp3-плеер за \$200-300 выбрать?
Hifiman HM-601 меня устраивал целиком...

\myhrule{}

Нужно проиндексировать пачку текстов. Потом сделать поиск по ним. Желателен простенький query-язык.
Какую поискать легковесную библиотеку для этого?
Желательно Питон или С++.

\myhrule{}

Как инсталлировать и запускать Cyc?

\myhrule{}

Как инсталлировать VMware Remote Console 10.0.4 на Ubuntu 19? Инсталлятор просто молча останавливается. Это известный симптом?

Или что вы используете для запуска VMware Workstation-машин на удаленном хосте на Ubuntu?

\myhrule{}

Win32-процесс А запущен.
Процесс Б аттачится к нему как отладчик, либо открывает его используя OpenProcess().
ReadProcessMemory() работает, но не работает, если пытается читать незакоммиченные страницы процесса А.

Проблема: как заставить менеджер памяти Windows закоммитить страницу в процессе А из userland-а процесса Б?
Я могу всунуть в процесс А инструкцию чтения, запустить её, страница закоммитится, но это не решение.

\myhrule{}

Если знаете что-то, пожалуйста помогите мне: \EMAIL{}, Telegram: @yurichev, Skype: dennis.yurichev
}
\EN{% To translators: don't bother to translate this... english-only version.

\begin{center}
\LARGE{} This is my own bulletin board \normalsize{}
\end{center}

\textbf{This book is probably outdated already}.
(Unless it was just downloaded from \url{https://beginners.re/}.)

The book is \href{https://github.com/DennisYurichev/RE-for-beginners/commits/master}{changing too often},
content being added, bugs are (hopefully) being fixed.
The latest version is always at \url{https://beginners.re/}.

This PDF you currently reading was compiled at \today{}.

\myhrule{}

If you have printed this book on paper, can you please send me a picture of it, for collection?\\
\EMAIL{}, Telegram: @yurichev.

\myhrule{}

My dear readers! From time to time, I have questions, I don't know who (or where) to ask.
Or I'm just lazy...
Can you please help me?

\myhrule{}

Can anyone help me with CBMC? I have a question.

\myhrule{}

What HiFi mp3-player for \$200-300 is good for its money?
I was happy with Hifiman HM-601...

\myhrule{}

A pack of texts are to be indexed. Then a search is required. A simple query-language is desirable.
What lightweight library would you recommend?
Preferably Python or C++.

\myhrule{}

How to install and run Cyc?

\myhrule{}

How do you install VMware Remote Console 10.0.4 on Ubuntu 19? It just suddenly exits during installation.
Is it known symptom?

Or what do you use to run VMware Workstation VMs on remote Ubuntu box?

\myhrule{}

A win32 process A is running.
Process B is attaching to it as a debugger, or opens it using OpenProcess().
ReadProcessMemory() works OK, but fails if it tries to read uncommited memory pages of process A.

The problem: how to force the Windows Memory Manager to commit a page in process A from userland of process B?
I can inject a read instruction into process A, run it, and the page would be committed, but this is not the solution.

\myhrule{}

If you know something, please help me: \EMAIL{}, Telegram: @yurichev, Skype: dennis.yurichev

}
%\DE{\input{1st_page_DE}}
%\ES{\input{1st_page_ES}}
%\CN{\input{1st_page_CN}}

\iffalse
\RU{\input{dedication_RU}}
\EN{\input{dedication_EN}}
\FR{\input{dedication_FR}}
\JA{\input{dedication_JA}}
\DE{\input{dedication_DE}}
\IT{\input{dedication_IT}}
\fi

\input{page_after_cover}
\input{call_for_translators}

\shorttoc{%
    \RU{Краткое оглавление}%
    \EN{Abridged contents}%
    \ES{Contenidos abreviados}%
    \PTBRph{}%
    \DE{Inhaltsverzeichnis (gekürzt)}%
    \PLph{}%
    \IT{Sommario}%
    \THAph{}\NLph{}%
    \FR{Contenus abrégés}%
    \JA{簡略版}
    \TR{İçindekiler}
}{0}

\tableofcontents
\cleardoublepage

\cleardoublepage
\input{preface}

\mainmatter

\input{parts}

\EN{\input{appendix/appendix}}
\RU{\input{appendix/appendix}}
\DE{\input{appendix/appendix}}
\FR{\input{appendix/appendix}}
\IT{\input{appendix/appendix}}
\input{acronyms}

\bookmarksetup{startatroot}

\clearpage
\phantomsection
\addcontentsline{toc}{chapter}{%
    \RU{Глоссарий}%
    \EN{Glossary}%
    \ES{Glosario}%
    \PTBRph{}%
    \DE{Glossar}%
    \PLph{}%
    \IT{Glossario}%
    \THAph{}\NLph{}%
    \FR{Glossaire}%
    \JA{用語}
    \TR{Bolum}
}
\printglossaries

\clearpage
\phantomsection
\printindex

\end{document}
